\nonstopmode{}
\documentclass[letterpaper]{book}
\usepackage[times,inconsolata,hyper]{Rd}
\usepackage{makeidx}
\usepackage[utf8,latin1]{inputenc}
% \usepackage{graphicx} % @USE GRAPHICX@
\makeindex{}
\begin{document}
\chapter*{}
\begin{center}
{\textbf{\huge Package `gmum.r'}}
\par\bigskip{\large \today}
\end{center}
\begin{description}
\raggedright{}
\item[Type]\AsIs{Package}
\item[Title]\AsIs{Package with models proposed by GMUM group at Jagiellonian
University}
\item[Version]\AsIs{1.0}
\item[Date]\AsIs{2014-02-25}
\item[Author]\AsIs{Stanislaw Jastrzebski}
\item[Maintainer]\AsIs{Stanislaw Jastrzebski }\email{grimghil@gmail.com}\AsIs{}
\item[Description]\AsIs{More about what it does (maybe more than one line)}
\item[License]\AsIs{MIT}
\item[Depends]\AsIs{Rcpp (>= 0.10.4)}
\item[LinkingTo]\AsIs{Rcpp}
\item[NeedsCompilation]\AsIs{yes}
\end{description}
\Rdcontents{\R{} topics documented:}
\inputencoding{utf8}
\HeaderA{gmum-utils-package}{What the package does (short line)}{gmum.Rdash.utils.Rdash.package}
\aliasA{gmum-utils}{gmum-utils-package}{gmum.Rdash.utils}
\keyword{package}{gmum-utils-package}
%
\begin{Description}\relax
More about what it does (maybe more than one line)
\textasciitilde{}\textasciitilde{} A concise (1-5 lines) description of the package \textasciitilde{}\textasciitilde{}
\end{Description}
%
\begin{Details}\relax

\Tabular{ll}{
Package: & gmum-utils\\{}
Type: & Package\\{}
Version: & 1.0\\{}
Date: & 2014-02-25\\{}
License: & What license is it under?\\{}
}
\textasciitilde{}\textasciitilde{} An overview of how to use the package, including the most important \textasciitilde{}\textasciitilde{}
\textasciitilde{}\textasciitilde{} functions \textasciitilde{}\textasciitilde{}
\end{Details}
%
\begin{Author}\relax
Who wrote it

Maintainer: Who to complain to <yourfault@somewhere.net>
\end{Author}
%
\begin{References}\relax
\textasciitilde{}\textasciitilde{} Literature or other references for background information \textasciitilde{}\textasciitilde{}
\end{References}
%
\begin{SeeAlso}\relax
\textasciitilde{}\textasciitilde{} Optional links to other man pages, e.g. \textasciitilde{}\textasciitilde{}
\textasciitilde{}\textasciitilde{} \code{\LinkA{<pkg>}{<pkg>}} \textasciitilde{}\textasciitilde{}
\end{SeeAlso}
\inputencoding{utf8}
\HeaderA{hello\_gmum}{Description Hello gmum call! Details Prints out "Hello Gmum"}{hello.Rul.gmum}
%
\begin{Description}\relax
Description Hello gmum call! Details Prints out "Hello
Gmum"
\end{Description}
%
\begin{Usage}
\begin{verbatim}
hello_gmum()
\end{verbatim}
\end{Usage}
\inputencoding{utf8}
\HeaderA{rcpp\_hello\_world}{Simple function using Rcpp}{rcpp.Rul.hello.Rul.world}
%
\begin{Description}\relax
Simple function using Rcpp
\end{Description}
%
\begin{Usage}
\begin{verbatim}
rcpp_hello_world()	
\end{verbatim}
\end{Usage}
%
\begin{Examples}
\begin{ExampleCode}
## Not run: 
rcpp_hello_world()

## End(Not run)
\end{ExampleCode}
\end{Examples}
\inputencoding{utf8}
\HeaderA{test\_flow}{Description Testing architecture flow Details On success it should print out the argument.}{test.Rul.flow}
%
\begin{Description}\relax
Description Testing architecture flow Details On success it
should print out the argument.
\end{Description}
%
\begin{Usage}
\begin{verbatim}
test_flow(msg)
\end{verbatim}
\end{Usage}
\inputencoding{utf8}
\HeaderA{test\_libsvm}{Description Hello gmum call! Details Prints out "Hello Gmum"}{test.Rul.libsvm}
%
\begin{Description}\relax
Description Hello gmum call! Details Prints out "Hello
Gmum"
\end{Description}
%
\begin{Usage}
\begin{verbatim}
test_libsvm()
\end{verbatim}
\end{Usage}
\printindex{}
\end{document}
